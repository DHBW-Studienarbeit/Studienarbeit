\documentclass[../main.tex]{subfiles}

\begin{document}
\section{Faltungsoperation}
\section{Pooling}
Die Erzeugung von vielen Featuremaps aus einer einzelnen Eingabe gehört zu den grundlegenden Eigenschaften der im vorherigen Abschnitt beschriebenen Faltungsoperation. Dies führt allerdings zu einer erheblichen Vergrößerung der Datenmenge, die von nachfolgenden Layern bearbeitet werden muss. Um das Ausmaß dieser Vergrößerung einschätzen zu können, wird an dieser Stelle die Größe des Outputs einer Faltung in Abhängigkeit von der Eingabegröße und weiterer Eigenschaften der Faltungsschicht untersucht. 
In den nachfolgenden Gleichungen steht \(S\) für die Größe einer einzelnen Ein- bzw. Ausgabe. Die Variablen \(x\), \(y\) und \(f\) stehen für die Dimensionen der Ein- bzw. Ausgabe. Die Größe des Filterkernels ist durch \(F_x\) und \(F_y\) bestimmt, für die Schrittweite wird der Wert \(1\) angenommen. 
\begin{equation}
\begin{split}
S_{in} = x_{in} y_{in} f_{in}\\
S_{out} = x_{out} y_{out} f_{out}\\
x_{out} = x_{in} - F_x + 1\\
y_{out} = y_{in} - F_y + 1
\end{split}
\end{equation}
Unter der Bedingung, dass \(x_{in} >> F_x\) und \(y_{in} >> F_y\) ergibt sich die Näherung \(\frac{S_{out}}{S_{in}} = \frac{f_{out}}{f_{in}}\). Die Anzahl der Neuronen einer Faltungsschicht gegenüber der des vorherigen Layers steigt bei der Faltungsoperation demzufolge annähernd proportional zur Anzahl der Featuremaps. 
Mit der Datenmenge im Ausgang der Faltungsschicht steigt auch die Zahl der Aktivierungen, die in den nachfolgenden Schichten bearbeitet werden müssen. Dies stellt einen beträchtlichen Mehraufwand dar und kann zu einer signifikanten Steigerung der Rechenzeit und des Speicherbedarfs führen. Um diesen Mehraufwand zu reduzieren, sind die Faltungsschichten in den meisten Implementierungen von CNNs von sogenannten Pooling-Schichten gefolgt. Beim Pooling wird jede Featuremap in rechteckige Abschnitte unterteilt, die dann jeweils auf einen neuen Aktivierungswert abgebildet werden. Für diese Abbildung gibt es mehrere Methoden, einige davon werden im folgenden Abschnitt näher erläutert. 
\subsection{Average pooling}
TODO: Inhalt von \cite{paperMixedPooling} kann auf mehrere Arten zitiert werden \parencite[368]{paperMixedPooling}
\subsection{Max pooling}

\subsection{Vergleich}

\section{Anpassungen der Backpropagation}

\end{document}
\documentclass[../main.tex]{subfiles}

\begin{document}
Künstliche Intelligenz hält in immer mehr Bereichen des alltäglichen Lebens Einzug. Sprachassistenten auf dem Handy sind seit einigen Jahren selbst in Einsteigermodellen vorhanden und auch die Anzahl der Haushalten mit smarten Lautsprechern wächst \cite{statisticSprachassistenten}. Aktuelle Autos besitzen unter anderem einen Spurhalteassistenten und bremsen automatisch, falls ein Fußgänger die Fahrbahn betritt. Auf einem Tablett kann mit dem Finger auf den Bildschirm geschrieben werden woraufhin das Gerät die geschriebenen Buchstaben erkennt und den Text aus Unicode Zeichen schreibt. \par

Für einen Menschen ist es nicht schwierig, eine Zahl auf einem Bild richtig einzustufen oder zu entscheiden, ob ein Bild eine Person zeigt oder nicht. Für einen Computer ist dieser Entscheidungsvorgang jedoch sehr schwierig. {W. Hilberg} schreibt dazu: "Die meisten Menschen „rechnen“ nicht, sondern sie „denken“. Rechnen ist nur eine spezielle Variante des Denkens"\cite{articleKuenstlichesGehirn}. Computer beherrschen also nur einen Teil des Denkens, in diesem Bereich sind sie jedoch sehr gut, Berechnungen mit vielen, großen Zahlen sind für einen PC sehr einfach. \par

Um eine künstliche Intelligenz zu erschaffen, kann deswegen nicht die klassische Programmierung verwendet werden. Stattdessen werden sogenannte Neuronale Netzwerke verwendet. Diese Bestehen aus vielen Neuronen und verhalten sich ähnlich wie Neuronen innerhalb des Gehirns von Lebewesen. \par
Die Neuronen in einem Gehirn empfangen Signale in Form von elektrischen Impulsen und geben diese weiter an andere Neuronen. Dieses banale Prinzip ermöglicht uns, Dinge zu erlernen und gelerntes anzuwenden. Die Anzahl der Neuronen ist hierfür entscheidend, im menschlichen Gehirn existieren laut älteren Studien etwa 100 Milliarden Neuronen (vgl. \cite{articleKuenstlichesGehirn}). Nach aktuelleren Studien geht man heute von circa 86 Milliarden Neuronen aus \cite{articleWieVieleNervenzellenHatDasGehirn}. \par

Der Mensch erweitert im Laufe seines Lebens durch Interaktion mit der Umwelt sein Wissen. Genauso lernt ein Neuronales Netzwerk durch viele Trainingsdatensätze und Zyklen das gewünschte Verhalten. Wenn das Verhalten den Anforderungen entspricht, kann das Trainieren abgebrochen werden. Das trainierte Netzwerk mit den gelernten Werten kann dann für den Anwendungsfall verwendet werden. Durch diese Methode können beliebige Entscheidungsvorgänge trainiert werden, solange geeignete Trainingsdaten vorhanden sind. \par

Für die Bilderkennung werden spezielle, sogenannte {Convolutional Neuronal Networks} verwendet. Diese eignen sich Muster an, mit denen die Bilder abgescannt werden. Solch ein Netzwerk soll im Laufe dieser Arbeit entstehen, um Handgeschriebene Ziffern richtig zu erkennen. Dabei sollen drei verschiedene Implementierungen entstehen, die jeweils das Netzwerk auf eine spezielle Zielhardware optimiert. \par

\end{document}
\documentclass[../main.tex]{subfiles}

\begin{document}
Künstliche Intelligenz hält in immer mehr Bereichen des alltäglichen Lebens Einzug. Sprachassistenten auf dem Handy sind seit einigen Jahren selbst in Einsteigermodellen vorhanden und auch die Anzahl der Haushalten mit smarten Lautsprechern wächst. Aktuelle Autos besitzen unter anderem einen Spurhalteassistenten und bremsen automatisch, falls ein Fußgänger die Fahrbahn betritt. Auf einem Tablett kann mit dem Finger auf den Bildschirm geschrieben werden woraufhin das Gerät die geschriebenen Buchstaben erkennt und den Text aus Unicode Zeichen schreibt.

Für einen Menschen ist es nicht schwierig, eine Zahl auf einem Bild richtig einzustufen oder zu entscheiden, ob ein Bild eine Person zeigt oder nicht. Für einen Computer ist dieser Entscheidungsvorgang jedoch sehr schwierig. 
Im Gegensatz dazu sind Berechnungen mit vielen, großen Zahlen für einen PC sehr einfach, währenddessen sich Personen damit sehr schwer tun. "Die meisten Menschen „rechnen“ nicht, sondern sie „denken“. Rechnen ist nur eine spezielle Variante des Denkens"(\cite{articleKuenstlichesGehirn}).

Für diese Art der Problemlösung kann nicht die klassische Programmierung verwendet werden. Stattdessen werden sogenannte Neuronale Netzwerke verwendet. Dieses Bestehen aus vielen Neuronen und verhalten sich ähnlich wie das Gehirn von Lebewesen. 
Der Mensch erweitert im Laufe seines Lebens durch Interaktion mit der Umwelt sein Wissen. Genauso lernt ein Neuronales Netzwerk durch viele Trainingsdatensätze und Zyklen das gewünschte Verhalten. Wenn das Verhalten den Anforderungen entspricht, kann das Trainieren abgebrochen werden. Das trainierte Netzwerk mit den gelernten Werten kann nun für den Anwendungsfall verwendet werden. Durch diese Methode können beliebige Entscheidungsvorgänge trainiert werden, solange geeignete Trainingsdaten vorhanden sind.

Für die Bilderkennung werden spezielle, sogenannte {Convolutional Neuronal Networks} verwendet. Diese eignen sich Muster an, mit denen die Bilder abgescannt werden. Solch ein Netzwerk soll im Laufe dieser Arbeit entstehen, um Handgeschriebene Ziffern richtig zu erkennen. Dabei sollen drei verschiedene Implementierungen entstehen, die jeweils das Netzwerk auf eine spezielle Zielhardware optimiert.

\end{document}
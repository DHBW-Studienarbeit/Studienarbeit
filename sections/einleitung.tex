\documentclass[../main.tex]{subfiles}

\begin{document}
Künstliche Intelligenz hält in immer mehr Bereichen des alltäglichen Lebens Einzug. Sprachassistenten auf dem Handy sind seit einigen Jahren selbst in Einsteigermodellen vorhanden und auch die Anzahl der Haushalten mit smarten Lautsprechern wächst. Aktuelle Autos besitzen unter anderem einen Spurhalteassistenten und bremsen automatisch, falls ein Fußgänger die Fahrbahn betritt. Bei Tabletts kann mit dem Finger auf den Bildschirm geschrieben werden und das Gerät erkennt die Buchstaben und schreibt den Text aus normalen Zeichen.

Für einen Menschen ist es nicht schwierig, eine Zahl auf einem Bild richtig einzuschätzen oder zu entscheiden, ob ein Bild eine Person zeigt oder nicht. Für einen Computer ist dieser Entscheidungsvorgang jedoch sehr schwierig.
Im Gegensatz dazu sind Berechnungen mit vielen, großen Zahlen für einen PC sehr einfach, währenddessen sich Personen damit sehr schwer tun.

Damit ein Computer diesen Entscheidungsvorgang trotzdem lösen kann, muss eine andere Vorgehensweise als die klassische Programmierung verwendet werden. Für diese Probleme werden sogenannte Neuronale Netzwerke verwendet. Dieses Bestehen aus vielen Neuronen und verhalten sich ähnlich wie das Gehirn von Lebewesen. Der Mensch häuft im Laufe seines Lebens durch Interaktion mit der Umwelt sein Wissen an. Genauso lernt ein Neuronales Netzwerk durch viele Trainingsdatensätze das gewünschte Verhalten an. Durch diese Methode können beliebige Entscheidungsvorgänge trainiert werden, solange geeignete Trainingsdaten vorhanden sind.

Für die Bilderkennung werden sogenannte {Convolutional Neuronal Networks} verwendet. Diese eignen sich Muster an, mit denen die Bilder abgescannt werden. Solch ein Netzwerk soll im Laufe dieser Arbeit entstehen, um Handgeschriebene Ziffern richtig zu erkennen.

\end{document}
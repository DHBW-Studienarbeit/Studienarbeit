\documentclass[../main.tex]{subfiles}

\begin{document}
\begin{otherlanguage}{english}
This thesis deals with the development of convolutional neuronal networks. Within the context of this thesis, four different implementations were developed. The first implementation functions as a template of the three different implementations. There is no optimization of any sort inside and runs fully serial.

The other implementations were optimized for a specific hardware. On one hand, there is one implementation for x86 processors. These are common in Office-PCs and Notebooks. In the other hand, there is a implementation for CUDA-GPUs. This in optimized for graphic cards from Nvidia. Another implementation was designed for servers with Intel Xeon Phi Co-processors. The Xeon Phi is a CPU-PCIe-Card with over 50 kernels inside.

For testing purpose, the MNIST data set was used. This in an data set with 65000 images of handwritten numbers. 55000 are for testing purpose and 10000 for verification purpose.

At the end of this thesis, there is a comparison between the implementation of this piece of work and TensorFlow, a library designed to create neuronal networks which is used in many services by Google.
\end{otherlanguage}
\end{document}